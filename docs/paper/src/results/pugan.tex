\section{PUGANを用いた構造提案} \label{sec:result_pugan}

本節ではPUGANを適用した際の結果を示す。

PUGANを適用して構造生成を行い、生成された構造が活性を持つ確率の分布や、生成された構造の骨格について検討した。

\subsection{使用したアーキテクチャ}

\tbref{tb:pugan_parameter_search}にPUGANのアーキテクチャで検討したパラメータを示す。
\begin{table}[tbp]
\centering
\caption{PUGANで検討したパラメータ} \label{tb:pugan_parameter_search}
\begin{tabular}{lll}\hline
& 項目 & パラメータ候補 \\\hline
層の種類 & Generatorの入力層 & Dense \\
& Generatorの隠れ層 & LSTM, GRU \\
& Generatorの出力層 & Softmax \\
& Discriminatorの入力層 & Embedding \\
& Discriminatorの隠れ層 & Dense \\
& Discriminatorの出力層 & Softmax \\\hline
各層のパラメータ & Embeddingの次元数 & 16 \\
& Denseの次元数 & 32, 64 \\
& LSTMの次元数 & 16, 32, 64 \\
& GRUの次元数 & 16, 32, 64 \\
& Conv1Dのチャネル数 & 16, 32, 64 \\
& Conv1Dのフィルタサイズ & 9, 10 \\
& 活性化関数 & ReLU, SELU, Tanh \\\hline
最適化 & バッチサイズ & 512 \\
& 最適化手法 & SGD, Adam \\
& 学習率 & 5e-4, 2.5e-4, 1e-4 \\\hline
\end{tabular}
\end{table}

\tbref{tb:pugan_model_detail}に採用されたパラメータを示す。
最適化手法はPositive GeneratorとNegative GeneratorにはAdamを利用し、学習率は\SI{5e-4}を採用した。
Positive DiscriminatorとUnlabeled Discriminator、Negative Discriminatorの3つにはSGDを利用し、
学習率は\SI{2.5e-4}を採用した。
GANに基づくモデルではGeneratorの学習率は高く、Discriminatorの学習率は低く設定することが多いが、
最適な学習率の組み合わせを5つのネットワークに対して調整するのは難しい。
そこで、AAEの学習において用いたパラメータを転用することで探索を省略した。
\begin{table}[tbp]
\centering
\caption{モデルのアーキテクチャ} \label{tb:pugan_model_detail}
\begin{tabular}{llclc}\hline
& 層の種類 & 層数 & パラメータ & 活性化関数 \\\hline
Positive Generator & Dense & 1 & 次元数64 & Tanh \\
& GRU & 3 & 次元数64 & ReLU \\
& Dense & 1 & 次元数32 & ReLU \\
& Dense & 1 & --- & Softmax \\\hline
Negative Generator & Dense & 1 & 次元数64 & Tanh \\
& GRU & 3 & 次元数64 & ReLU \\
& Dense & 1 & 次元数32 & ReLU \\
& Dense & 1 & --- & Softmax \\\hline
Positive Discriminator & Embedding & 1 & 次元数16 & --- \\
& Conv1D & 2 & チャネル数32, フィルタサイズ$[9,9]$ & ReLU \\
& Dense & 1 & 次元数32 & ReLU \\
& Dense & 1 & --- & Softmax \\\hline
Unlabled Discriminator & Embedding & 1 & 次元数16 & --- \\
& Conv1D & 2 & チャネル数32, フィルタサイズ$[9,9]$ & ReLU \\
& Dense & 1 & 次元数32 & ReLU \\
& Dense & 1 & --- & Softmax \\\hline
Negative Discriminator & Embedding & 1 & 次元数16 & --- \\
& Conv1D & 2 & チャネル数32, フィルタサイズ$[9,9]$ & ReLU \\
& Dense & 1 & 次元数32 & ReLU \\
& Dense & 1 & --- & Softmax \\\hline
\end{tabular}
\end{table}

\subsection{生成構造の評価方法}

生成された構造の評価には\secref{sec:result_aae}と同様、生成された構造が活性を持つ確率および生成された構造が持つ骨格を考慮した。

生成された構造が活性を持つ確率には\secref{sec:structure_check}と同様、Random Forestで構築した判別モデルの予測確率を利用した。
ハイパーパラメータも同様に回帰木の数は$900$とし、記述子にはMorgan Fingerprintを利用した。
Morgan Fingerprintの半径は$2$、ビット数は$512$とした。

生成構造の構造式としての評価は、検証用データとして訓練には用いていない既知医薬品が持つ分子骨格と比較することで行った。

\subsection{結果と考察}

\subsubsection{生成構造の活性予測値分布}

\figref{fig:pugan_prob_hist}にPositive GeneratorとNegative Generatorが生成した構造の
活性予測値のヒストグラムを示した。横軸が活性のある確率を、縦軸が正規化された度数を表す。
\figref{fig:aae_prob_hist}と比較するとAAEに比べPUGANを適用した場合は生成された構造群の間に大きな違いがないことが分かった。
\begin{figure}[tbp]
    \centering
    \includegraphics[width=0.75\columnwidth]{../resource/pugan/prob_hist.png}
    \caption{PUGANで生成された構造の活性予測値分布} \label{fig:pugan_prob_hist}
\end{figure}

\subsubsection{生成構造の例}

\figref{fig:pugan_identical}にPUGANの生成構造のうち、検証構造と同一の構造を生成できたものを示す。
\figref{fig:pugan_similar}にPUGANの生成構造のうち、検証構造と類似する骨格を持っているものの一部を示す。
PUGANを利用した場合も、医薬品類似の化合物を生成できることが分かった。
\begin{figure}[tbp]
    \centering
    \includegraphics[width=0.4\columnwidth]{../resource/pugan/match1.png}
    \caption{検証構造と同一の構造を生成できたパターン} \label{fig:pugan_identical}
\end{figure}
\begin{figure}[tbp]
    \begin{minipage}[b]{0.49\hsize}
    \centering
    \includegraphics[width=0.8\columnwidth]{../resource/pugan/sim1_valid.png} 
    \end{minipage}
    \begin{minipage}[b]{0.49\hsize}    
    \centering
    \includegraphics[width=0.9\columnwidth]{../resource/pugan/sim1_new.png}
    \end{minipage}

    \begin{minipage}[b]{0.49\hsize}    
        \centering
        \includegraphics[width=0.8\columnwidth]{../resource/pugan/sim2_valid.png}
        \subcaption{検証構造}
    \end{minipage}
    \begin{minipage}[b]{0.49\hsize}    
        \centering
        \includegraphics[width=0.8\columnwidth]{../resource/pugan/sim2_new.png}
        \subcaption{生成構造}
    \end{minipage}
\caption{検証構造と同様の骨格を持った構造を生成できたパターン} \label{fig:pugan_similar}
\end{figure}

\subsection{課題と解決の方針}

\figref{fig:pugan_strange}にPUGANの生成構造のうち、適切でないと思われる構造の例を示す。
これらの構造は活性が高い構造によく含まれる部分構造を骨格に複数付加することによって生成していると考えられる。
\begin{figure}[tbp]
    \begin{minipage}[b]{1\hsize}
        \centering
        \includegraphics[width=0.6\columnwidth]{../resource/pugan/invalid1.png}
    \end{minipage}
    \begin{minipage}[b]{1\hsize}    
        \centering
        \includegraphics[width=0.6\columnwidth]{../resource/pugan/invalid2.png}
    \end{minipage}
\caption{適切でないと思われる構造の例} \label{fig:pugan_strange}
\end{figure}

このように適切でないと思われる構造が生成される原因として考えられるのは、Discriminatorが弱いということである。
GANを元にした手法を学習する際は学習が容易なDiscriminatorを意図的に弱く設計することでスムーズに学習を進めることが多い。
しかし、今回の場合のように同じ置換基が大量についている構造を分類するためにはDiscriminatorを少し強く設計しても良いかもしれない。
