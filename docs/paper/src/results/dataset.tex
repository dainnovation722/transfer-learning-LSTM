\section{データセット}

データセットとしてChEMBL22とADRA2Aを用いた(\tbref{tb:dataset})。

ChEMBL22データセットはChEMBLデータベース\cite{chembl22}より取得した$1,300,000$件余りのデータから
ランダムに$500,000$件をサンプリングしてきた、活性値情報を持たないデータである。
M1モデルに対して教師なし学習を行うのに用い、SMILESから化学構造一般の特徴を得た。
ADRA2AデータセットはChEMBLデータベースより抽出された
ヒトalpha-2A adrenergic receptorに対する阻害活性についてのデータであり、$501$件存在する。
M2モデルに対してChEMBL22と合わせて半教師付き学習を行って活性値の特徴を得るのに用いた。
\begin{table}[tbp]
    \centering
    \caption{用いたデータセット} \label{tb:dataset}
    \begin{tabular}{lrcc}\hline
      名称 & データ数 & 活性値情報 & 活性\\\hline
      ChEMBL 22 & $500,000$ & × & --- \\
      ADRA2A & $501$ & ○ & あり \\\hline
    \end{tabular}
\end{table}

\subsection{データセットの分割}

ADRA2Aのデータセットを訓練$401$構造、検証$100$構造に分割した。
検証構造はモデルの訓練には用いず、最終的に生成された構造が検証構造と類似の骨格を持っているかを確認するのに用いた。

\subsection{行った前処理}

前処理として、キラリティ情報の削除、SMILESのカノニカライズ、原子数の制約、SMILES文字列長の制限、重複の削除を行った。
以下に詳細を示す。

\subsubsection{キラリティ情報の削除}

本来、薬理活性においてキラリティは重要な役割を持つ。
しかし、ChEMBLやADRA2Aのデータセットには
本来キラリティを考慮しなければならないのにキラリティ情報が含まれていない分子が存在する。

このような状況においてキラリティ情報を利用することはモデルの学習に悪影響を及ぼす。
そこで、本研究では全分子からキラリティの情報を削除することで対応した。

\subsubsection{SMILESのカノニカライズ}

SMILES表記は開始原子の位置や環を辿る順番などによって複数の表記が存在しうる。
そこで、本研究ではSMILESを一意に定めるカノニカライズを施した。
カノニカライズにはフリーのケモインフォマティクスライブラリであるRDKit\cite{rdkit}を用いた。

\subsubsection{原子数の制約}

SMILESの文字列を生成させる上で、小さすぎる分子はノイズとなる。
そこで、水素以外の原子が10個未満の分子はデータセットから除外した。
これは先行研究\cite{Blaschke2018}で用いられていた基準と同様である。

\subsubsection{SMILES文字列長の制限}

本研究では、SMILESを生成させる際にRNNを利用した。
RNNは多少の系列長の変化に対しては頑健なものの、極端に長い系列や極端に短い系列に対しては対応が難しい。
\figref{fig:smiles_length}に各データセットにおけるSMILESの文字列長のヒストグラムを示す。
横軸がSMILESでの文字列長、縦軸が度数である。
横軸で表示されている範囲が該当する文字列長のSMILESが存在する範囲である。
ADRA2Aが$120$文字までにすべての分子が収まっているのに対し、ChEMBL22には極端に文字列長が長い分子が含まれていることが分かる。
そこで、本研究ではADRA2Aに合わせ、ChEMBL22のうち$120$文字以下の分子についてのみ利用することとした。
\begin{figure}[tbp]
    \begin{minipage}[h]{1\hsize}
    \centering
    \includegraphics[width=0.9\columnwidth]{../resource/preprocessing/chembl_smiles_length.png} 
    \subcaption{ChEMBL22} \label{fig:chembl_smiles_length}
    \end{minipage}
    \begin{minipage}[h]{1\hsize}    
    \centering
    \includegraphics[width=0.9\columnwidth]{../resource/preprocessing/adra2a_smiles_length.png}
    \subcaption{ADRA2A} \label{fig:adra2a_smiles_length}
    \end{minipage}
\caption{各データセットにおけるSMILES文字列長} \label{fig:smiles_length}
\end{figure}

\subsubsection{重複の削除}

ChEMBL22中に存在するADRA2Aの構造をすべて削除した。
