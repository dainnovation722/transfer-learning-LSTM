\chapter{結論}

\section{まとめ}

大量の教師なしデータと少量の教師付きデータを組み合わせた半教師付き学習を化学データに適用した。
深層生成モデルにはVAE、AAE、PUGANの3種類を用いた。

\secref{sec:result_vae}ではVAEを適用した。
活性値の回帰を行い、Random Forestでの教師付き学習と比較して良好な予測を行うことができた。
一方でVAEで構造生成を行った場合は炭素鎖を伸ばしたり末端の原子の置換を行ったりなど、本質的でない構造変化を施した構造が多く生成されることが分かった。
これは活性を実数値で評価したことで、許容される活性値の変化が小さくなってしまったことに原因があると考えられる。

\secref{sec:result_aae}ではAAEを適用した。
潜在変数の可視化によって、ChEMBL22の分布にくらべてADRA2Aのデータセットを限られた領域に押し込めることができ、
その付近からサンプリングしてきた構造はより活性を持ちやすいことを示した。
これは化学空間全体を探索するのに比べ、提案手法を利用すると効率的な医薬品構造が設計できることを意味する。
構造生成では検証構造(訓練データには含んでいないが、医薬品と判明している構造)と完全に同一もしくは同様の骨格を持っている構造を生成することができた。
提案手法が新規医薬品構造を提案する能力を持っていることを示した。

\secref{sec:result_pugan}ではPUGANを適用した。
PUGANはラベルが付いていないデータを「活性がある可能性もあるし、ない可能性もある」と明示的に扱うことができるため、
半教師付き学習として有用であることを期待していた。
しかし、生成された構造にはSMILESとして妥当でない構造が多く含まれ、うまく学習が進んでいないことが示唆された。
原因としてはPUGANのモデルの複雑さが考えられる。
PUGANは2つのGeneratorと3つのDiscriminatorが含まれる複雑なモデルであり、それらを協調して学習を進めなければならない。
そのため、細かいパラメータの変更やデータセットの性質によって大きく学習が不安定になってしまう。
今回の検証の範囲ではAAEを上回る結果を得ることはできなかった。

\secref{sec:result_total}では各手法の比較を行った。
SMILESとして妥当な構造が生成された割合、活性の予測値が0.5を越えた構造の割合、検証用データと全く同じ構造を生成した割合を比較し、
AAEが最も良い性能を発揮することを示した。

\section{今後の展望}

今後の展望としては、生成される構造の多様性を増すことが挙げられる。
本研究ではデータが少ない状況でも利用可能な手法を提案した。しかし、活性値情報付きのデータが少ないと
活性を失わずに構造変化させるパターンを学習することが難しい。
この問題に対処するには、他のタンパクに対する活性値情報付きのデータセットを併せて利用し、
マルチモーダル的に学習を行うことが考えられる。
学習できる構造変化のパターンを増やすことで生成構造の多様性を増すことができると期待される。

また、今回は構造生成をSMILESを生成することで行った。SMILESを生成するにはモデルにSMILESの文法を学習させる必要がある。
一方で近年ではグラフを直接生成する手法も提案されている。グラフの生成モデルを化学データに適用した場合についても検討を行うべきであろう。