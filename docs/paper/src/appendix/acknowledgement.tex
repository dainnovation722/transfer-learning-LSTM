\chapter*{謝辞}
本研究のディレクションにあたって指導をいただきました船津公人教授に心より感謝いたします。
談話会などで先生の質疑に答えているうちにするすると研究の方向性がまとまっていく感覚はなかなか心地よいものでした。
講義や講演などでお忙しいなか、本当にありがとうございました。

小寺正明准教授には主に隔週のミーティングでアドバイスを多くいただきました。
実験者としての視点も併せて伝えていただいたことで一層のブラッシュアップへとつながったと感じています。
修士2年になってからのアカデミックな文章の指導は先生が中心となってくださり、たいへんためになりました。

田中健一助教には日々のディスカッションでお世話になりました。
各種統計手法に始まり、研究の進め方、まとめ方、伝え方に至るまで。
学部4年生からの3年間で多くのことを学ばせていただきました。

秘書の岡村さんには日々の生活を支えてくださったことに対して感謝の意を述べたいと思います。
何度も何度も「ホワイトボードのマーカーください」とお願いに行ってすみませんでした。
ホワイトボードのマーカーは箱でストックしていただけると幸いです。

船津・小寺研究室の研究員、学生のみなさま、本当にありがとうございました。
先輩方には考えをぶつける楽しさを、後輩たちには知識を伝える喜びを、それぞれ学ぶことができました。
この研究室で過ごした3年間は私の宝物です。

同期の菅原くん、井上くん、見内くん、Seolさん、天野くん、松本くん、菱沼さん、楽しかったです。
4月以降もまたどこかで会いましょう。
菅原くんは4月からもよろしく。

最後に、私の歩む道を静かに見守り、支えてくださった両親に深く感謝の意を示して、謝辞とさせていただきます。
